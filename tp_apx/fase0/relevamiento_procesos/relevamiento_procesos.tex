\subsection{Definición de los procesos y necesidades de información}

% Proceso 'PVenta'
\subsubsection{Proceso \textsf{PVenta}}

\begin{apx1}
{Depto. de Ventas}
{Jefe de ventas}
{PVenta}
{07/08/2007}
{1}
{Gestión de clientes, facturación y generación de información estadística sobre ventas.}
  \apxUnoItem{Atender consulta del cliente.}{Consultar solvencia del cliente.}
  \apxUnoItem{Tomar pedido del cliente.}{Consultar existencia de stock.}
  \apxUnoItem{Consultar precios.}{Facturar.}
  \apxUnoItem{}{Generar reporte mensual de ventas.}
\end{apx1}

\begin{apx2}
{Depto. de Ventas}
{Jefe de ventas}
{PVenta}
{07/08/2007}
{1}
  \apxDosItem{Factura}{Logística, Ventas, Cliente}{Diario}{}
  \apxDosItem{Nota de crédito}{Logística, Ventas, Cliente}{Diario}{}
  \apxDosItem{Nota de débito}{Logística, Ventas, Cliente}{Diario}{}
\end{apx2}

\begin{apx3}
{Depto. de Ventas}
{Jefe de ventas}
{PVenta}
{07/08/2007}
{1}
  \apxTresItem{Nota de pedido}{Cliente}{Diario}{}
  \apxTresItem{Estado de cuenta corriente}{Finanzas}{Mensual}{}
  \apxTresItem{Padrón de clientes}{Finanzas}{Mensual}{}
  \apxTresItem{Clientes suspendidos}{Finanzas}{Semanal}{}
  \apxTresItem{Informe de estado de stock}{Logística}{Diario}{}
  \apxTresItem{Remito}{Logística}{Diario}{}
  \apxTresItem{Factura}{Ventas}{Diario}{}
  \apxTresItem{Nota de crédito}{Ventas}{Diario}{}
  \apxTresItem{Nota de débito}{Ventas}{Diario}{}
\end{apx3}

\begin{apx4}
{Depto. de Ventas}
{Jefe de ventas}
{PVenta}
{07/08/2007}
{1}
  \apxCuatroItem{Estado de stock en tránsito}{Logística}{Diario}
    {Se refiere al stock de una sucursal que todavía no está presente en ella, pero que ya se ha
     pedido y está en camino.}
  \apxCuatroItem{Estado de stock en tiempo real}{Logística}{Diario}
    {La sucursal interesada en vender productos de otra, consulta el stock actual de la misma en
     tiempo real.}
\end{apx4}


% Proceso 'PControlStock'
\subsubsection{Proceso \textsf{PControlStock}}

\begin{apx1}
{Depto. de Logística}
{Jefe de logística}
{PControlStock}
{07/08/2007}
{1}
{Control, distribución y mantenimiento de stock.}
  \apxUnoItem{Controlar el stock actual de la sucursal local.}
    {Emitir pedido al departamento de compras.}
  \apxUnoItem{Realizar entrega de stock a sucursal.}
    {Generar informe de estado de stock.}
  \apxUnoItem{Recibir mercaderías.}{Almacenar mercaderías.}
\end{apx1}

\begin{apx2}
{Depto. de Logística}
{Jefe de logística}
{PControlStock}
{07/08/2007}
{1}
  \apxDosItem{Pedido de compras}{Compras}{Semanal}
    {Documento interno para solicitar al departamento de Compras que realice el pedido a los proveedores correspondientes.}
  \apxDosItem{Informe de estado de stock}{Ventas}{Diario}{}
  \apxDosItem{Remito}{Logística, Ventas, Cliente}{Diario}{}
\end{apx2}

\begin{apx3}
{Depto. de Logística}
{Jefe de logística}
{PControlStock}
{07/08/2007}
{1}
  \apxTresItem{Informe de compras}{Compras}{Diario}
    {Con esta información del Depto. de Compras, el área de logística
     es capaz de clasificar el stock de una sucursal como ``stock en
     tránsito''.}
  \apxTresItem{Factura}{Ventas}{Diario}{}
  \apxTresItem{Remito}{Logística}{Diario}{}
  \apxTresItem{Remito}{Proveedor}{Semanal}{}
  \apxTresItem{Nota de crédito}{Ventas}{Diario}{}
  \apxTresItem{Nota de débito}{Ventas}{Diario}{}
\end{apx3}

\begin{apx4}
{Depto. de Logística}
{Jefe de logística}
{PControlStock}
{07/08/2007}
{1}
  \apxCuatroItem{Estado de stock en tránsito}{Logística (de las demás sucursales)}{Diario}
    {El área de logística de cada sucursal recibe información sobre el stock
     en tránsito de las demás sucursales.}
  \apxCuatroItem{Estado de stock en tiempo real}{Logística (de las demás sucursales)}{Diario}
    {El área de logística de cada sucursal recibe información sobre el stock
     las demás sucursales, en tiempo real.}
\end{apx4}


% Proceso 'PCompra'
\subsubsection{Proceso \textsf{PCompra}}

\begin{apx1}
{Depto. de Compras}
{Jefe de compras}
{PCompra}
{07/08/2007}
{1}
{A partir de una orden de pedido, se solicita a proveedores el envío de mercaderías.}
  \apxUnoItem{Recibir pedido de compras.}{Emitir pedido de proveedores.}
  \apxUnoItem{}{Realizar pagos a proveedores.}
\end{apx1}

\begin{apx2}
{Depto. de Compras}
{Jefe de compras}
{PCompra}
{07/08/2007}
{1}
  \apxDosItem{Informe de compras}{Logística}{Semanal}{}
  \apxDosItem{Pedido a proveedores}{Compras, Proveedor}{Semanal}{}
\end{apx2}

\begin{apx3}
{Depto. de Compras}
{Jefe de compras}
{PCompra}
{07/08/2007}
{1}
  \apxTresItem{Pedido de compras}{Logística}{Semanal}{}
  \apxTresItem{Factura}{Proveedor}{Semanal}{}
  \apxTresItem{Pedido a proveedores}{Compras}{Semanal}{}
\end{apx3}

\begin{apx4}
{Depto. de Compras}
{Jefe de compras}
{PCompra}
{07/08/2007}
{1}
  \apxCuatroItem{\textit{No hay ítems en este APX para esta área}}{}{}{}
\end{apx4}

