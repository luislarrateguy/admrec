\subsection{Plan de acción}

\begin{proyecto}
{1}
{Evaluación, planificación y adquisición de software y hardware.}
{Analista en Sistemas César Sandrigo.}

  \begin{etapas}
    \etapa{Definir la estructura de procesamiento de datos para las áreas involucradas.}
    \etapa{Establecer el hardware y software que soporta la estructura definida.}
    \etapa{Definir la compra del equipamiento para llamado a licitación de proveedores.}
    \etapa{Estudiar las propuestas y ofertas de los proveedores.}
    \etapa{Decidir por alguna de las alternativas.}
    \etapa{Realizar las pruebas de benchmark.}
    \etapa{Efectuar la adquisición del hardware y el software.}
    \etapa{Planificar la implementación y puesta a punto.}
  \end{etapas}

  \begin{recursos}
    \recurso{Humanos: Personal involucrado.}
    \recurso{Financieros.}
  \end{recursos}

  \tiempo{3 meses a partir de aprobado el proyecto.}
\end{proyecto}


\begin{proyecto}
{2}
{Desarrollo del nuevo sistema de manejo de stock en tiempo real.}
{Milton Pividori}

  \begin{etapas}
    \etapa{Relevar requisitos funcionales del nuevo sistema.}
	\etapa{Realizar el analisis del sistema.}
	\etapa{Estudiar y evaluar las diferentes tecnologías existentes en el mercado.}
	\etapa{Elegir una o mas tecnologías.}
	\etapa{Realizar el diseño segun las metodologías aplicables a la/las tecnología/s en cuestión.}
	\etapa{Desarrollar el nuevo sistema.}
	\etapa{Realizar pruebas de funcionalidad.}
	\etapa{Llevar a cabo pruebas de facilidad de uso y de alcance de objetivos.}
	\etapa{Corregir errores si es necesario.}
	\etapa{Planificar migración al nuevo sistema.}	
  \end{etapas}

  \begin{recursos}
    \recurso{Humanos: Programadores, Ingenieros en sistemas.}
	\recurso{Financieros: para adquicisión de herramientas.}
  \end{recursos}

  \tiempo{8 meses.}
\end{proyecto}


\begin{proyecto}
{3}
{Capacitación del personal en el uso del nuevo sistema.}
{Luis Larrateguy}

  \begin{etapas}
    \etapa{Capacitar a los empleados de Ventas en el nuevo workflow para ventas
      desde una sucursal con entrega en otra.}
    \etapa{Capacitar a los empleados de Logistica para poder responder a las
      consultas del sector venta de cualquier sucursal.}
    \etapa{Capacitar a los empleados de Compras para que sigan los 
      procedimientos de registro de pedidos.}
    \etapa{Capacitar a lso diseñadores de publicidad impresa y de otros medios
      en el workflow de Ventas remoto.}
  \end{etapas}

  \begin{recursos}
    \recurso{Humanos: Ventas, Compras, Logistica, Publicistas}
    \recurso{Financiero}
  \end{recursos}

  \tiempo{1 a 2 meses}
\end{proyecto}


\begin{proyecto}
{4}
{Actualización de la infraestructura de comunicación en todas las sucursales}
{Milton Pividori}

  \begin{etapas}
    \etapa{Relevar requerimientos de funcionamiento ideales.}
	\etapa{Estudiar las diferentes tecnologías.}
	\etapa{Evaluar la disponibilidad geografica de las diferentes tecnologías.}
	\etapa{Estudiar la oferta de los diferentes proveedores.}
	\etapa{Elegir una de las tecnologías propuestas.}
	\etapa{Realizar contratos con los proveedores de las tecnologías elegidas.}
	\etapa{Realizar evaluación de performance y alcance de requisitos.}
  \end{etapas}

  \begin{recursos}
    \recurso{Humanos: Ingenieros en sistemas.}
	\recurso{Financieros.}
  \end{recursos}

  \tiempo{5 meses.}
\end{proyecto}


\begin{proyecto}
{5}
{Migrar desde el viejo sistema al nuevo.}
{César Sandrigo}

  \begin{etapas}
    \etapa{Capacitar al personal en el uso del nuevo sistema.}
	\etapa{Establecer mesas de ayuda.}
	\etapa{Realizar pruebas de operaciones con el nuevo sistema.}
	\etapa{Realizar migraciones de progresivas de distintas funcionalidades.}
	\etapa{Retirar de funcionamiento al viejo sistema definitivamente.}
	\etapa{Realizar auditorías para evaluar el funcionamiento del nuevo sistema.}
  \end{etapas}

  \begin{recursos}
    \recurso{Humanos.}
  \end{recursos}

  \tiempo{2 meses, a partir de terminado el desarrollo del sistema.}
\end{proyecto}