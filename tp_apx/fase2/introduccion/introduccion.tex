\section{Sobre Red Megatone}


% Introducción
\subsection{Introducción}

La Red no es una empresa unitaria sino que está formada por tres empresas y una
entidad dependiente: Electrónica Megatone (con sede en la ciudad de Santa Fe),
Carsa (sede en Resistencia), Bazar Avenida (casa central en Rafaela) y Confina.

Electrónica Megatone nació hace unos 30 años. Su creador, actual presidente de
la Red, comenzó reparando electrodomésticos y con el paso del tiempo fue
conociendo el mercado, hasta que decidió comprar y revender electrodomésticos.
Así se fue consolidando Electrónica Megatone y poco a poco convirtiéndose en
una importante empresa local. 

Unos 12 años después la empresa se asocia con Carsa y Bazar Avenida, formando
así lo que es en la actualidad Red Megatone. Desde ese momento la Red fue
tomando más fuerza y ganando mercado. Esta fusión produjo un gran cambio en la
estructura de las organizaciones y la reestructuración trajo la creación,
aproximadamente dos o tres años después, de Confina.

Confina nace como una separación de la empresa, encargándose, en un principio,
de la administración de las cuentas corrientes de la Red, y luego brindando
préstamos personales al público en general. Confina tiene una importante labor
dentro de las finanzas de la empresa ya que se transformó en una importante
financiera en la región.

Así, en la actualidad, cuenta con una dotación de aproximadamente 1500
empleados distribuidos en 128 sucursales a lo largo del país, y ocupa el
segundo puesto en ventas de artículos para el hogar.


% Estructura organizacional
\subsection{Estructura organizacional}

\begin{figure}[h]
  \incluirimagen{width=\textwidth}{sobre_red_megatone/organigrama.pdf}
  \caption{Organigrama de Red Megatone}
  \label{fig:organigrama}
\end{figure}

El organigrama de la figura~\ref{fig:organigrama} revela la estructura
funcional de la empresa.  Claramente se nota una diferenciación horizontal,
aunque también se observa una centralización  por la existencia de la Gerencia
General y el Comité Ejecutivo, que así reduce la función del presidente a la de
un veedor mayor con posibilidad de toma de decisiones.


% Problemática detectada
\subsection{Problemática detectada}

Al estudiar el funcionamiento del centro de cómputo detectamos un problema en
el sistema de stock de la empresa. Para profundizar sobre el problema antes
debemos describir el funcionamiento del sistema de stock actual, así también
como el equipamiento con el que se cuenta.


\subsubsection{Equipamiento}

El Centro de Cómputos, que administra todo el Sistema de Gestión Integral,
cuenta con la siguiente infraestructura:

\begin{itemize}

  \item Existe un Servidor Central (Sun-multiprocesador), muy potente, que
    nuclea toda la información que se maneja en el sistema.

  \item Hay otros Servidores Locales más pequeños de marca Intel, también
    multiprocesador, que gestionan la operatoria de, como promedio, 5 sucursales
    ubicadas dentro de una misma zona geográfica. Suman un total de aproximadamente. 25
    servidores.

  \item Cada sucursal cuenta con un equipo por usuario, con impresora de carro
    ancho (matriz de puntos) para emitir los documentos corrientes en caso de ser
    necesario. También se cuenta con impresoras láser para impresiones más
    importantes. Hay aproximadamente 500 PC’s en toda la empresa, sin contar los
    equipos que se utilizan como servidores.

  \item La Red cuenta además con 3 depósitos de mercaderías controlados por una
    administración central, la cual se encarga de gestionar el stock general de
    toda la Red. Cada depósito tiene como misión abastecer a un grupo de
    sucursales, las cuales tienen su propio stock (más reducido que el de los
    depósitos). Cabe aclarar que los depósitos son gestionados informáticamente
    directamente por el servidor central sin servidores intermedios lo cual
    simplifica mucho llevar el stock de los depósitos.

\end{itemize}


\subsubsection{Funcionamiento del sistema en general}

Los servidores locales se conectan con el servidor central dos veces por día,
al mediodía luego del cierre, y a la noche después de finalizar la jornada,
para informar al servidor central la operatoria reciente del grupo de
sucursales que cada uno tiene a su cargo. Luego se procede a transferir a cada
sucursal el stock existente en las sucursales restantes de todo el país y otros
datos, y poder así empezar al día siguiente con el stock ``actualizado'', ya
que cada sucursal puede, además de vender productos en existencia de su propio
stock, vender los que se encuentran en stock de otras sucursales.


\subsubsection{Funcionamiento general del sistema de stock actual. Planteo del
problema}

En general, el sistema de stock funciona de la siguiente manera: al recibir el
remito del proveedor, se registra una entrada de mercaderías. Cuando se vende,
la factura ocasiona una salida de stock, pero también puede ocurrir que el
cliente devuelva las mercaderías, entonces las notas de créditos vuelven a
ocasionar una entrada de ellas. Además se puede dar de baja mercadería porque
se echó a perder por algún otro motivo.

Para controlar este flujo de mercadería se utiliza un sistema, el cual se
encarga de llevar el stock de las sucursales de todo el país. Su funcionamiento
es complejo pero trataremos de simplificarlo para así poder dar a entender el
problema. 

Cuando los servidores locales se conectan con el servidor central dos veces por
día para informar a éste la operatoria reciente del grupo de sucursales que
cada uno tiene a su cargo se envían, entre otros datos, el stock actual de cada
sucursal. Aquí el inconveniente: cada sucursal, tiene actualizado solamente el
stock de las sucursales que comparten el mismo servidor local, mientras que de
las otras sucursales (las que pertenecen a otro servidor local) se tiene
información de la última actualización del servidor central. Y como se puede
vender productos que están en el stock de otras sucursales se genera un
conflicto al intentar vender un producto que se encuentra en otra sucursal
lejana, ya que puede ocurrir que justamente ese artículo ya haya sido vendido
en esa sucursal y, como todavía no se actualizó, figure como existencia cuando
no es así.

Como lo planteado anteriormente no resulta sencillo de entender (según nuestro
criterio), proponemos un ejemplo:

Supongamos que, antes del mediodía, en la sucursal Red Megatone de Reconquista
un cliente (podría ser turista) decide comprar un televisor poco común en el
mercado, con la condición de que ese televisor sea entregado en la ciudad de
Mendoza. El empleado que gestiona la venta, se dirige al stock de la sucursal
de Mendoza (actualizado la noche anterior), y ve que hay en existencia un solo
televisor del solicitado por el cliente. Con esta información, el empleado
realiza la venta del televisor y se compromete a entregarlo en la cuidad de
Mendoza. Cuando este empleado informa a la sucursal de Mendoza para que estos
realicen la entrega, ellos le comunican que el televisor que el solicita para
la entrega, fue vendido durante la mañana, por lo cual el cliente de
Reconquista tendrá que esperar a que llegue al stock de la sucursal de Mendoza
el televisor en cuestión. Además se producirá un error de datos cuando el
servidor central procese los datos ya que habrá dos ventas (una en Mendoza y
otra Reconquista) para el mismo artículo.

\pagebreak
