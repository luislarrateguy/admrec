\section{Sobre Red Megatone}

% Introducción
\subsection{Introducción}
La Red no es una empresa unitaria sino que está formada por tres empresas y una
entidad dependiente: Electrónica Megatone (con sede en la ciudad de Santa Fe),
Carsa (sede en Resistencia), Bazar Avenida (casa central en Rafaela) y Confina.

Electrónica Megatone nació hace unos 30 años. Su creador, actual presidente de
la Red, comenzó reparando electrodomésticos y con el paso del tiempo fue
conociendo el mercado, hasta que decidió comprar y revender electrodomésticos.
Así se fue consolidando Electrónica Megatone y poco a poco convirtiéndose en
una importante empresa local. 

Unos 12 años después la empresa se asocia con Carsa y Bazar Avenida, formando
así lo que es en la actualidad Red Megatone. Desde ese momento la Red fue
tomando más fuerza y ganando mercado. Esta fusión produjo un gran cambio en la
estructura de las organizaciones y la reestructuración trajo la creación,
aproximadamente dos o tres años después, de Confina.

Confina nace como una separación de la empresa, encargándose, en un principio,
de la administración de las cuentas corrientes de la Red, y luego brindando
préstamos personales al público en general. Confina tiene una importante labor
dentro de las finanzas de la empresa ya que se transformó en una importante
financiera en la región.

Así, en la actualidad, cuenta con una dotación de aprox. 1500 empleados
distribuidos en 128 sucursales a lo largo del país, y ocupa el segundo puesto
en ventas de artículos para el hogar.


% Estructura organizacional
\subsection{Estructura organizacional}
\begin{figure}[h]
  \incluirimagen{width=\textwidth}{sobre_red_megatone/organigrama.png}
  \caption{Organigrama de Red Megatone}
  \label{fig:organigrama}
\end{figure}

El organigrama de la figura~\ref{fig:organigrama} revela la estructura
funcional de la empresa.  Claramente se nota una diferenciación horizontal,
aunque también se observa una centralización  por la existencia de la Gerencia
General y el Comité Ejecutivo, que así reduce la función del presidente a la de
un veedor mayor con posibilidad de toma de decisiones.

En el organigrama figura el desarrollo solamente de Sistemas sobre el cuál
desarrollaremos la problemática. Dentro del sector, se dispone de tres Centros
de Cómputo: el primero que gestiona los datos de Bazar Avenida que se encuentra
en Rafaela; el segundo, que atiende el sistema de Confina; y el tercero, que se
encarga de Megatone y de Carsa (ubicado en la ciudad de Santa Fe). Sobre este
último profundizaremos el estudio.

La anterior disposición del manejo de la información hace que las tres grandes
zonas adopten un sistema federal, es decir que tienen casi total autonomía para
tomar decisiones operativas, mientras que las decisiones estratégicas se toman
entre los directores y el presidente.

\TODO{Falta todo lo otro}


\begin{apx}{Ventas}{Jefe ventas}{PVentas}{9 de agosto}{1}{Este es el depto. de ventas esto probando a ver que pasa con el ancho e la tabla}
\end{apx}
