%
% Realizado con Software Libre
% 
% Usen esta página para hacerles saber a los lectores de su documento,
% que el mismo esta desarrollado íntegramente con Software Libre
%
% Versión 1.0.1 - 23/Septiembre/2006
% Por Milton Pividori

\thispagestyle{empty}

\begin{center}
\LARGE{Hecho con Software Libre}
\end{center}

\noindent

El trabajo práctico, así como este documento, fueron realizados íntegramente
usando Software Libre. Las herramientas utilizadas fueron:

\begin{flushleft}
\begin{itemize}

% LaTeX
\item \textbf{\LaTeX} - \href{http://www.latex-project.org/}
  {http://www.latex-project.org/}
\linebreak\LaTeX{} es un sistema de preparación de documentos. Éste
  documento esta hecho con él.

% Subversion
\item \textbf{Subversion} - \href{http://subversion.tigris.org/}
  {http://subversion.tigris.org/}
\linebreak Subversion es un sistema de control de versiones diseñado
  específicamente para reemplazar a CVS. Lo utilizamos para versionar
  tanto el código fuente como la documentación.

% Ubuntu Linux
\item \textbf{Ubuntu Linux} - \href{http://www.ubuntu.com/}
  {http://www.ubuntu.com/}
\linebreak Basada en Debian GNU/Linux, Ubuntu es un sistema operativo
enteramente de fuente abierta.

% GNU Aspell
\item \textbf{GNU Aspell} - \href{http://aspell.sourceforge.net//}
  {http://aspell.sourceforge.net/}
\linebreak GNU Aspell es una utilidad para chequear la ortografía.

% Vim
\item \textbf{Vim} - \href{http://www.vim.org/}
  {http://www.vim.org/}
\linebreak Vi IMproved es un editor de texto avanzado. Hay disponible una
versión gráfica para Windows.

% Mono
\item \textbf{Mono} - \href{http://www.mono-project.com/}
  {http://www.mono-project.com/}
\linebreak Mono es una implementación libre del framework .NET de Microsoft.
Provee el software necesario para desarrollar y correr aplicaciones clientes y
servidoras en GNU/Linux, Solaris, Mac OS X, Windows y Unix. Está impulsado por
Novell.

% MonoDevelop
\item \textbf{MonoDevelop} - \href{http://www.monodevelop.com/}
  {http://www.monodevelop.com/}
\linebreak IDE libre para GNU/Linux diseñado para C\# y otros lenguajes
de .NET.

\end{itemize}
\end{flushleft}

\begin{center}
  \includegraphics[scale=0.50]{hechoconsl/logofsf.jpg}
  \linebreak\href{http://www.fsf.org}{http://www.fsf.org}
\end{center}

