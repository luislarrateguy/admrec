\documentclass{beamer}

\mode<presentation>
{
  \usetheme{Warsaw}
  \setbeamercovered{transparent}
}

\usepackage[spanish]{babel}
\usepackage[utf8]{inputenc}
\usepackage{times}
\usepackage[T1]{fontenc}

\title[Proyecto Mono]
{Proyecto Mono: Implementación libre y multiplataforma de .NET}

%\subtitle{}

\date[TP Adm. de Recursos]
{Trabajo práctico de investigación y desarrollo \\ Administración de Recursos}

\subject{Trabajo práctico de investigación y desarrollo - Administración de Recursos - 2007}


\begin{document}

\begin{frame}
  \titlepage
\end{frame}

\begin{frame}
  \frametitle{Contenido}
  \tableofcontents
\end{frame}


% % % % % % % % % % % % % % % % % % % % % % % % % % % % % % % % % % % % % % % %
\section{Nuestros objetivos}
% % % % % % % % % % % % % % % % % % % % % % % % % % % % % % % % % % % % % % % %

\begin{frame}{Nuestros objetivos con el Trabajo Práctico}
  \begin{itemize}
    \item Brindar como profesionales soluciones basadas en Software Libre.
    \item Investigar Mono como una plataforma de desarrollo para alcanzar lo anterior.
    \item Mostrarles que con Software Libre se puede lograr un desarrollo serio.
    \item REVEER ESTOS OBJETIVOS ENTRE TODOS.
  \end{itemize}
\end{frame}


% % % % % % % % % % % % % % % % % % % % % % % % % % % % % % % % % % % % % % % %
\section{Introducción a .NET}
% % % % % % % % % % % % % % % % % % % % % % % % % % % % % % % % % % % % % % % %

% % % % % % % %
\subsection{Características principales}
% % % % % % % %

\begin{frame}{Ventajas}
  \begin{itemize}
    \item Interoperabilidad (COM, P/Invoke).
    \item Motor común de ejecución (CIL, JIT).
    \item Independencia del lenguaje (CTS). COMPLETAR
    \item Librería de clases base (BCL): Disponible para todos los lenguajes.
    \item Despliegue simplificado (GAC).
    \item Seguridad (diferentes niveles de seguridad).
  \end{itemize}
\end{frame}


\begin{frame}{Desventajas}
  \begin{itemize}
    \item \alert{Sólo disponible para Windows.} Mono es multiplataforma.
    \item \alert{Código cerrado.} Mono es libre.
    \item \alert{Las sucesivas versiones van dejando de lado a los SOs viejos.}
      Mono es libre y multiplataforma.
  \end{itemize}
\end{frame}


% % % % % % % %
\subsection{Conceptos básicos}
% % % % % % % %

\begin{frame}%{.NET Framework}
  % - Eine Überschrift fasst einen Rahmen verständlich zusammen. Man
  %   muss sie verstehen können, selbst wenn man nicht den Rest des
  %   Rahmens versteht.

  \begin{itemize}
    \item CLI (Common Language Infrastructure)
      \begin{itemize}
        \item Componente más importante.
        \item Especificación abierta.
        \item Describe el código ejecutable y el entorno de ejecución.
        \item Permite que la plataforma sea multilenguaje.
      \end{itemize}
  \end{itemize}
\end{frame}

\begin{frame}[plain]
  \begin{centering}
    \pgfimage[height=\paperheight]{cli}
  \end{centering}
\end{frame}

\begin{frame}
  \begin{itemize}
    \item CLR (Common Language Runtime)
      \begin{itemize}
        \item Implementación de CLI de Microsoft.
        \item CLS (Common Language Specification). Todos los lenguajes deben
          cumplirla.
      \end{itemize}

    \item CIL: Como el bytecode de Java.
    \item BCL (Base Class Library)
    \item CLR + BCL = .NET Framework

    \item Assemblies (ensamblados)
      \begin{itemize}
        \item Unidad de ejecución.
        \item Albergan el código CIL.
        \item Dos tipos: procesos (EXE) y librerías (DLL)
      \end{itemize}
  \end{itemize}
\end{frame}


% % % % % % % %
\subsection{Código de ejemplo en C\#}
% % % % % % % %

\begin{frame}
  \begin{itemize}
    \item ejemplo de código c-sharp
  \end{itemize}
\end{frame}


% % % % % % % %
\subsection{Estandarización y licenciamiento}
% % % % % % % %

\begin{frame}{CLI y C\#}
  \begin{itemize}
    \item ECMA estandars en 2001.
    \item ISO estandars en 2003.
    \item Implica patentes disponibles sin regalías.
    \item Sin embargo no se aplica a lo que no esté dentro del estandar,
      como Windows Forms, ADO.NET y ASP.NET.
  \end{itemize}
\end{frame}


% % % % % % % % % % % % % % % % % % % % % % % % % % % % % % % % % % % % % % % %
\section{Mono}
% % % % % % % % % % % % % % % % % % % % % % % % % % % % % % % % % % % % % % % %

% % % % % % % %
\subsection{Cuestiones interesantes}
% % % % % % % %

\begin{frame}{Mono}
  \begin{itemize}
    \item Soportado por Novell.
    \item Altamente portable:
    \begin{itemize}
      \item Familia Unix: GNU/Linux, Mac OS X, Solaris, BSDs\ldots
      \item Familia Windows: NT, 2000, XP.
      \item Sistemas embebidos (Maemo).
    \end{itemize}

    \item Arquitecturas de CPU:
    \begin{itemize}
      \item x86, PowerPC, AMD64, Sparc, s390, IA64, ARM
      \item 64 bits no soportados en todos los sistemas operativos.
    \end{itemize}

    \item Activamente desarrollado (datos 2006).
    \begin{itemize}
      \item Más de 300 contribuidores.
      \item Más de 20 desarrolladores full time.
    \end{itemize}
  \end{itemize}
\end{frame}

\begin{frame}[plain]
  \begin{centering}
    \pgfimage[width=\textwidth]{mono_virtual_plataform}
  \end{centering}
\end{frame}


% % % % % % % %
\subsection{Objetivos de Mono}
% % % % % % % %

\begin{frame}
  \frametitle{Objetivos de Mono}
  \framesubtitle{Palabras de Miguel De Icaza (fundador del proyecto)}

  \begin{itemize}
    \item Originalmente, mejorar nuestra plataforma de desarrollo en Linux.
    \item En tanto la comunidad crecía, crecer para soportar las APIs de Microsoft.

    \item En tanto Mono se volvía mas completo:
      \begin{itemize}
        \item Proveer de un runtime multiplataforma completo.
        \item Permitir a los desarrolladores de Windows portar a Linux.
      \end{itemize}
  \end{itemize}
\end{frame}


% % % % % % % %
\subsection{Estado actual del proyecto}
% % % % % % % %

\begin{frame}[plain]
  \frametitle{Estado actual de Mono}
  \framesubtitle{Stack de Mono}

  \begin{centering}
    \pgfimage[width=\textwidth]{mono_stacks}
  \end{centering}
\end{frame}

\begin{frame}[plain]{Con respecto a MS .NET}
  \begin{centering}
    \pgfimage[width=\textwidth]{api_space}
  \end{centering}

  C\# 3.0, Silverlight\ldots
\end{frame}


% % % % % % % %
\subsection{¿Quién usa Mono?}
% % % % % % % %

\begin{frame}
  \begin{itemize}
    \item algo
  \end{itemize}
\end{frame}


% % % % % % % %
\subsection{MonoDevelop}
% % % % % % % %

\begin{frame}
  \begin{itemize}
    \item algo
  \end{itemize}
\end{frame}


% % % % % % % %
\subsection{MoMa}
% % % % % % % %

\begin{frame}
  \begin{itemize}
    \item algo
  \end{itemize}
\end{frame}


% % % % % % % % % % % % % % % % % % % % % %
\appendix
\section<presentation>*{\appendixname}
\subsection<presentation>*{Referencias}

\begin{frame}[allowframebreaks]
  \begin{thebibliography}{10}
    
  \beamertemplatearticlebibitems

  \bibitem{Wikipedia}
    Wikipedia
    \newblock {\em .NET Framework}

  \bibitem{Icaza}
    Miguel De Icaza
    \newblock {\em Mono Meeting} Octubre 2006
  \end{thebibliography}
\end{frame}

\end{document}


