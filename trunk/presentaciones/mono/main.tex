% El contenido de ésta presentación está bajo la licencia Creative Commons
% License Attribution-ShareAlike 2.5 Argentina.

\documentclass{beamer}

\mode<presentation>
{
  \usetheme{Warsaw}
  \setbeamercovered{transparent}
}

\usepackage[spanish]{babel}
\usepackage[utf8]{inputenc}
\usepackage{times}
\usepackage[T1]{fontenc}
\usepackage{verbatim}
\usepackage{lgrind}

\title[Proyecto Mono]
{Proyecto Mono: Implementación libre y multiplataforma de .NET}

%\subtitle{}

\date[TP Adm. de Recursos]
{Trabajo práctico de investigación y desarrollo \\ Administración de Recursos}

\subject{Trabajo práctico de investigación y desarrollo - Administración de Recursos - 2007}


\begin{document}

\begin{frame}
  \titlepage
\end{frame}

\begin{frame}
  \frametitle{Contenido}
  \begin{scriptsize}
  \tableofcontents
  \end{scriptsize}
\end{frame}


% % % % % % % % % % % % % % % % % % % % % % % % % % % % % % % % % % % % % % % %
\section{Nuestros objetivos}
% % % % % % % % % % % % % % % % % % % % % % % % % % % % % % % % % % % % % % % %

\begin{frame}{Nuestros objetivos con el Trabajo Práctico}
  \begin{itemize}
    \item Brindar como profesionales soluciones basadas en Software Libre.
    \item Investigar Mono como una plataforma de desarrollo para alcanzar lo anterior.
    \item Mostrarles que con Software Libre se puede lograr un desarrollo serio.
    \item REVEER ESTOS OBJETIVOS ENTRE TODOS.
  \end{itemize}
\end{frame}


% % % % % % % % % % % % % % % % % % % % % % % % % % % % % % % % % % % % % % % %
\section{Introducción a .NET}
% % % % % % % % % % % % % % % % % % % % % % % % % % % % % % % % % % % % % % % %

% % % % % % % %
\subsection{Características principales}
% % % % % % % %

\begin{frame}{Características}
  \begin{itemize}
    \item Compilación Just-in-time (JIT).
    \item Gestión de memoria (Garbage collector).
    \item Gestión de errores (excepciones).
    \item Multithreading.
  \end{itemize}
\end{frame}

\begin{frame}{Ventajas}
  \begin{itemize}
    \item Interoperabilidad (COM, P/Invoke).
    \item Motor común de ejecución (CIL, JIT).
    \item Independencia del lenguaje (CTS). COMPLETAR
    \item Librería de clases base (BCL): Disponible para todos los lenguajes.
    \item Despliegue simplificado (GAC).
    \item Seguridad (diferentes niveles de seguridad).
  \end{itemize}
\end{frame}


\begin{frame}{Desventajas}
  \begin{itemize}
    \item \alert{Sólo disponible para Windows.} Mono es multiplataforma.
    \item \alert{Código cerrado.} Mono es libre.
    \item \alert{Las sucesivas versiones van dejando de lado a los SOs viejos.}
      Mono es libre y multiplataforma.
  \end{itemize}
\end{frame}


% % % % % % % %
\subsection{Conceptos básicos}
% % % % % % % %

\begin{frame}%{.NET Framework}
  % - Eine Überschrift fasst einen Rahmen verständlich zusammen. Man
  %   muss sie verstehen können, selbst wenn man nicht den Rest des
  %   Rahmens versteht.

  \begin{itemize}
    \item CLI (Common Language Infrastructure)
      \begin{itemize}
        \item Componente más importante.
        \item Especificación abierta.
        \item Describe el código ejecutable y el entorno de ejecución.
        \item Permite que la plataforma sea multilenguaje.
      \end{itemize}
  \end{itemize}
\end{frame}

\begin{frame}[plain]
  \begin{centering}
    \pgfimage[height=\paperheight]{imagenes/cli}
  \end{centering}
\end{frame}

\begin{frame}
  \begin{itemize}
    \item CLR (Common Language Runtime)
      \begin{itemize}
        \item Implementación de CLI de Microsoft.
        \item CLS (Common Language Specification). Todos los lenguajes deben
          cumplirla.
      \end{itemize}

    \item CIL: Como el bytecode de Java.
    \item BCL (Base Class Library)
    \item CLR + BCL = .NET Framework

    \item Assemblies (ensamblados)
      \begin{itemize}
        \item Unidad de ejecución.
        \item Albergan el código CIL.
        \item Dos tipos: procesos (EXE) y librerías (DLL)
      \end{itemize}
  \end{itemize}
\end{frame}


% % % % % % % %
\subsection{Estandarización y licenciamiento}
% % % % % % % %

\begin{frame}{CLI y C\#}
  \begin{itemize}
    \item ECMA estandars en 2001.
    \item ISO estandars en 2003.
    \item Implica patentes disponibles sin regalías.
    \item Sin embargo no se aplica a lo que no esté dentro del estándar,
      como Windows Forms, ADO.NET y ASP.NET.
  \end{itemize}
\end{frame}


% % % % % % % %
\subsection{Características de C\#}
% % % % % % % %

\begin{frame}{C\#}
  \begin{exampleblock}{Propiedades (getter y setter en Java)}
    \verbatiminput{codigo/propiedades.cs}
  \end{exampleblock}
\end{frame}

\begin{frame}
  \begin{exampleblock}{Indexadores}
    \verbatiminput{codigo/indexers.cs}
  \end{exampleblock}
\end{frame}

\begin{frame}
  \begin{exampleblock}{Atributos}
    \verbatiminput{codigo/atributos.cs}
  \end{exampleblock}
\end{frame}

\begin{frame}
  \begin{exampleblock}{Iteradores}
    \begin{scriptsize}
      \verbatiminput{codigo/iteradores.cs}
    \end{scriptsize}
  \end{exampleblock}
\end{frame}

\begin{frame}
  \begin{exampleblock}{Clases parciales}
    \begin{scriptsize}
      \verbatiminput{codigo/clases_parciales.cs}
    \end{scriptsize}
  \end{exampleblock}
\end{frame}

\begin{frame}
  \begin{itemize}
    \item Delegados (como punteros a funciones).
  \end{itemize}

  \pause
  \begin{exampleblock}{Eventos}
    \begin{scriptsize}
      \verbatiminput{codigo/eventos.cs}
    \end{scriptsize}
  \end{exampleblock}
\end{frame}

\begin{frame}
  \begin{exampleblock}{Métodos anónimos}
    \begin{scriptsize}
      \verbatiminput{codigo/metodos_anonimos.cs}
    \end{scriptsize}
  \end{exampleblock}
\end{frame}

\begin{frame}
  \begin{exampleblock}{Sobrecarga de operadores}
    \begin{scriptsize}
      \verbatiminput{codigo/sobrecarga_operadores.cs}
    \end{scriptsize}
  \end{exampleblock}
\end{frame}

\begin{frame}
  \begin{exampleblock}{Tipos genéricos (Generics)}
    \begin{scriptsize}
      \verbatiminput{codigo/tipos_genericos.cs}
    \end{scriptsize}
  \end{exampleblock}
\end{frame}

\begin{frame}
  \begin{exampleblock}{Código inseguro (unsafe code)}
    \begin{scriptsize}
      \verbatiminput{codigo/codigo_inseguro.cs}
    \end{scriptsize}
  \end{exampleblock}
\end{frame}

\begin{frame}
  \frametitle{Futuro (C\# 3.0)}
  \framesubtitle{Algunas caracteristicas están soportadas por Mono}

  \begin{itemize}
    \item Variables locales de tipo implícito (keyword \emph{var}).
    \item Métodos extensores.
    \item Expresiones lambda.
    \item Inicializadores de objetos.
    \item Tipos anónimos.
    \item Arrays de tipo implícito.
    \item Expresiones de consulta.
    \item Árboles de expresiones.
  \end{itemize}
\end{frame}


% % % % % % % % % % % % % % % % % % % % % % % % % % % % % % % % % % % % % % % %
\section{Mono}
% % % % % % % % % % % % % % % % % % % % % % % % % % % % % % % % % % % % % % % %

% % % % % % % %
\subsection{Cuestiones interesantes}
% % % % % % % %

\begin{frame}{Mono}
  \begin{itemize}
    \item Soportado por Novell.
    \item Altamente portable:
    \begin{itemize}
      \item Familia Unix: GNU/Linux, Mac OS X, Solaris, BSDs\ldots
      \item Familia Windows: NT, 2000, XP.
      \item Sistemas embebidos (Maemo).
    \end{itemize}

    \item Arquitecturas de CPU:
    \begin{itemize}
      \item x86, PowerPC, AMD64, Sparc, s390, IA64, ARM
      \item 64 bits no soportados en todos los sistemas operativos.
    \end{itemize}

    \item Activamente desarrollado (datos 2006).
    \begin{itemize}
      \item Más de 300 contribuidores.
      \item Más de 20 desarrolladores full time.
    \end{itemize}
  \end{itemize}
\end{frame}

\begin{frame}[plain]
  \begin{centering}
    \pgfimage[width=\textwidth]{imagenes/mono_virtual_plataform}
  \end{centering}
\end{frame}


% % % % % % % %
\subsection{Objetivos de Mono}
% % % % % % % %

\begin{frame}
  \frametitle{Objetivos de Mono}
  \framesubtitle{Palabras de Miguel De Icaza (fundador del proyecto)}

  \begin{itemize}
    \item Originalmente, mejorar nuestra plataforma de desarrollo en Linux.
    \item En tanto la comunidad crecía, crecer para soportar las APIs de Microsoft.

    \item En tanto Mono se volvía mas completo:
      \begin{itemize}
        \item Proveer de un runtime multiplataforma completo.
        \item Permitir a los desarrolladores de Windows portar a Linux.
      \end{itemize}
  \end{itemize}
\end{frame}


% % % % % % % %
\subsection{Estado actual del proyecto}
% % % % % % % %

\begin{frame}[plain]
  \frametitle{Estado actual de Mono}
  \framesubtitle{Stack de Mono}

  \begin{centering}
    \pgfimage[width=\textwidth]{imagenes/mono_stacks}
  \end{centering}
\end{frame}

\begin{frame}[plain]{Con respecto a MS .NET}
  \begin{centering}
    \pgfimage[width=\textwidth]{imagenes/api_space}
  \end{centering}

  \vspace{2ex}
  C\# 3.0, Silverlight\ldots
\end{frame}


% % % % % % % %
\subsection{Herramientas, librerías\ldots}
% % % % % % % %

\begin{frame}{mcs: compilador de C\#}
  \begin{itemize}
    \item C\# escrito en C\#.
    \item Desde 2002 se compila a sí mismo (6 meses después de comenzar).
    \item Actualmente soporta C\# 2.0. VERIFICAR ESTO
    \item Posee algunas características de C\# 3.0 (que vendrá oficialmente con .NET 3.5)
  \end{itemize}
\end{frame}

\begin{frame}
  \begin{itemize}
    \item Compiladores y herramientas para utilizar varios lenguajes: Boo,
      Python, Ruby, Java (\alert{IKVM}), Visual Basic.NET, Prolog, PHP, JavaScript\ldots
    \item Toolkits gráficos: Windows.Forms 1.1 (y parte del 2.0), Gtk\#, Qyoto (Qt), Cocoa\# (MacOs X)
    \item OpenGL, Cairo, bases de datos, Gecko
    \item COMPLETAR
  \end{itemize}
\end{frame}


% % % % % % % %
\subsection{Licenciamiento}
% % % % % % % %

\begin{frame}{Licenciamiento}
  \begin{itemize}
    \item Compiladores: GPL
    \item Librerías: MIT X11
    \item Runtime: LGPL
    \item Novell retiene el copyright: licencia dual posible.
  \end{itemize}
\end{frame}


% % % % % % % %
\subsection{¿Quién usa Mono?}
% % % % % % % %

\begin{frame}{¿Quién usa Mono?}
  \begin{itemize}
    \item Novell.
    \item Otee: Desarrollan Unity, herramienta para el modelado 3D de juegos, que
      utiliza Mono en algunos componentes.
    \item Interopix: Corren Mono en un servidor FreeBSD con alta carga.
    \item Wikipedia: El indexado y búsqueda se lleva acabo por aplicaciones basadas en Mono.
    \item Aplicaciones: Banshee, Tomboy, iFolder, MonoDevelop, Beagle\ldots
    \item Muchos más ejemplos en la página oficial: \alert{www.mono-project.com}
  \end{itemize}
\end{frame}

\begin{frame}[plain]
  \frametitle{MP3/Wifi player}
  \framesubtitle{\ldots de Zing. Usan Mono para manejar el stack de aplicaciones.}

  \begin{centering}
    \pgfimage[width=\textwidth]{imagenes/sansa}
  \end{centering}
\end{frame}

\begin{frame}
  \frametitle{Zaspe\#}
  \framesubtitle{No todo es teoría\ldots}

  \begin{centering}
    \pgfimage[width=\textwidth]{imagenes/zaspe-sharp}
  \end{centering}
\end{frame}


% % % % % % % %
\subsection{MonoDevelop}
% % % % % % % %

\begin{frame}
  \frametitle{MonoDevelop: Un IDE para Mono}

  \begin{itemize}
    \begin{small}
      \item Es libre.
      \item Basado en SharpDevelop (el creador ahora esta en el equipo de MonoDevelop).
      \item Completado de código, sistema de plugins, refactoring, Gtk\# designer (Stetic)\ldots
      \item Proyectos C\#, Visual Basic.NET, Boo, Nemerle, Java (IKVM), ASP.NET\ldots
      \item Integración con Makefiles, Subversion, NUnit\ldots
      \item Abre soluciones hechas con Visual Studio 2003 y 2005.
      \item \alert{Todavía en fase de desarrollo. Última versión: 0.15}
    \end{small}
  \end{itemize}
\end{frame}


% % % % % % % %
\subsection{MoMa}
% % % % % % % %

\begin{frame}{Migración desde Windows}
  \begin{itemize}
    \item algo
  \end{itemize}
\end{frame}


% % % % % % % %
\subsection{Enlaces}
% % % % % % % %

\begin{frame}
  \begin{itemize}
    \item Mono Project: http://www.mono-project.com
    \item MonoDevelop: http://www.monodevelop.com
    \item Código fuente del TP: http://code.google.com/p/admrec/
  \end{itemize}
\end{frame}


% % % % % % % % % % % % % % % % % % % % % % % % % % % % % % % % % % % % % % % %
\appendix
\section<presentation>*{\appendixname}
% % % % % % % % % % % % % % % % % % % % % % % % % % % % % % % % % % % % % % % %

% % % % % % % %
\subsection<presentation>*{Referencias}
% % % % % % % %

\begin{frame}[allowframebreaks]
  \begin{thebibliography}{10}
    
  \beamertemplatearticlebibitems

  \bibitem{Wikipedia}
    Wikipedia
    \newblock {\em .NET Framework}

  \bibitem{Icaza}
    Microsoft
    \newblock {\em MSDN}

  \bibitem{Icaza}
    Miguel De Icaza
    \newblock {\em Mono Meeting} Octubre 2006 (Slides)

  \end{thebibliography}
\end{frame}


% % % % % % % %
\subsection<presentation>*{Licencia de la presentación}
% % % % % % % %

\begin{frame}{Licencia de esta presentación}
  \begin{tiny}
  \begin{itemize}
    \item Licencia: Creative Commons Atribución-CompartirDerivadasIgual 2.5 Argentina

    \item Usted es libre de:
      \begin{itemize}
      \begin{tiny}
        \item Copiar, distribuir, exhibir, y ejecutar la obra.
        \item Hacer obras derivadas.
      \end{tiny}
      \end{itemize}

    \item Bajo las siguientes condiciones:
      \begin{itemize}
      \begin{tiny}
        \item \textbf{Atribución}. Usted debe atribuir la obra en la forma
          especificada por el autor o el licenciante.
        \item \textbf{Compartir Obras Derivadas Igual}. Si usted altera,
          transforma, o crea sobre esta obra, sólo podrá distribuir la
          obra derivada resultante bajo una licencia idéntica a ésta.
      \end{tiny}
      \end{itemize}

    \item Ante cualquier reutilización o distribución, usted debe dejar
      claro a los otros los términos de la licencia de esta obra.
    \item Cualquiera de estas condiciones puede dispensarse si usted
      obtiene permiso del titular de los derechos de autor.
    \item Nada en esta licencia menoscaba o restringe los derechos
      morales del autor.
  \end{itemize}
  \end{tiny}
\end{frame}

\end{document}


