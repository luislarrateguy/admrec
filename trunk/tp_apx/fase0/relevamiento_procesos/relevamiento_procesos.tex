\subsection{Definición de los procesos y de sus necesidades de información}

% Proceso 'Venta'
\subsubsection{Proceso \textsf{Ventas}}

\begin{apx1}
{Depto. de Ventas}
{Jefe de ventas}
{Venta}
{07/08/2007}
{1}
{Gestión de clientes, facturación y generación de información estadística sobre ventas.}
  \apxUnoItem{Atender consulta del cliente.}{Consultar solvencia del cliente.}
  \apxUnoItem{Tomar pedido del cliente.}{Consultar existencia de stock.}
  \apxUnoItem{Consultar precios.}{Facturar.}
  \apxUnoItem{}{Generar reporte mensual de ventas.}
\end{apx1}

\begin{apx2}
{Depto. de Ventas}
{Jefe de ventas}
{Venta}
{07/08/2007}
{1}
  \apxDosItem{}{}{}{}
\end{apx2}


% Proceso 'ControlStock'
\subsubsection{Proceso \textsf{ControlStock}}
\begin{apx1}
{Depto. de Logística}
{Jefe de logística}
{ControlStock}
{07/08/2007}
{1}
{Control, distribución y mantenimiento de stock.}
  \apxUnoItem{Recibir pedido de sucursal.}{Emitir pedido al departamento de compras.}
  \apxUnoItem{Realizar entrega de stock a sucursal.}{Generar informe de estado de stock.}
\end{apx1}


% Proceso 'Compra'
\subsubsection{Proceso \textsf{Compra}}
\begin{apx1}
{Depto. de Compras}
{Jefe de compras}
{Compra}
{07/08/2007}
{1}
{A partir de una orden de pedido, se solicita a proveedores el envío de mercaderías.}
  \apxUnoItem{Recibir pedido de compras.}{Emitir pedido de proveedores.}
  \apxUnoItem{Recibir mercaderías.}{Almacenar mercaderías.}
\end{apx1}

