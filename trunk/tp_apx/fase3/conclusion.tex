\subsection{Conclusión}
Comparando ambos equipos, con las carácterísticas mencionadas, ambos se 
encuentran en similares condiciones.

El Equipo 1 (Athlon 64 3200+) presentó un mejor desempeño que el Equipo 2 
(Athlon 64 3000+) respecto a procesador y velocidad de trabajo en memoria,
el Equpio 2 presenta un mejor rendimiento en disco aunque no significativo
y mucho mejor desempeño en cuanto a Video.

Esto se debe a que utiliza una placa interna pero no incoporada a la placa
madre, como era el caso del Equipo 1.

Lo que beneficia al Equipo 1 es que para el uso de oficina que tendría, no
haría falta este desempeño en dibujo de polígonos y renderizados complejos
y se puede aprovechar para aumentar la potencia en otros aspectos.

En la comparación de las características físicas, en la mayoría está en 1er 
puesto el Equipo 1, y en los puntos en que no alcanza el primer lugar, puede 
solucinarse con la adquisición de periféricos extras.

Respecto a la garantía, el segundo equipo es el que se lleva el primer puesto, 
ya que ofrece una garantía extendida aparte de la proveida por los fabricantes 
de cada componente. Sin embargo el Equipo 1 tiene mejor precio y soporte 
técnico dentro de las 48 horas hábiles.

\begin{center}
\begin{tabular}{|lcc|} \hline
\footnotesize\textbf{Item} & \footnotesize\textbf{Equipo 1} & \footnotesize\textbf{Equipo 2} \\\hline
Soporte técnico & 1 & 2 \\\hline
Período de garantía & 2 & 1 \\\hline
Precio & 1 & 2 \\\hline
\end{tabular}
\end{center}

