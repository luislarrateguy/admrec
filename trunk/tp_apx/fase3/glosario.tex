\newpage
\subsection{Glosario}

\begin{itemize}

\item \textbf{Chipset}: Es un conjunto de circuitos integrados que se encarga 
de realizar las funciones que el microprocesador delega en ellos. En general
 se encargan de la comunicación entre procesador y los demás dispositivos 
componentes de la PC: Memoria (solo en procesadores Intel, los procesadores 
AMD traen intergrado el controlador de memoria), discos rígidos, tarjetas de 
expansión, puertos externos (como USB), etc.

\item \textbf{DDR2-SDRAM}(\textit{Double-Data-Rate-2 Synchronous Dynamic Random Access
Memory}): DDR2 es un tipo de memoria RAM. Forma parte de la familia SDRAM de
tecnologías de memoria de acceso aleatorio, que es una de las muchas
implementaciones de la DRAM. Los modulos DDR2 son capaces de trabajar con 4 bits
por ciclo, es decir 2 de ida y 2 de vuelta en un mismo ciclo mejorando
sustancialmente el ancho de banda potencial bajo la misma frecuencia de una DDR
tradicional (si una DDR a 200MHz reales entregaba 400MHz nominales, la DDR2 por
esos mismos 200MHz reales entrega 800MHz nominales).

\item \textbf{MDIPS}: (Millions of Dhrystone Instructions Per Second) acrónimo de
 ``millones de instrucciones por segundo''. Utiliza algoritmos propuestos
por el benchmark Dhrystone. Es una forma de medir la potencia de los
 procesadores. 

\item \textbf{MWIPS}: (Million Whetstones Instructions Per Second) acrónimo de
 ``millones de instrucciones por segundo''. Utiliza algoritmos y estructuras de datos
descriptos para el benchmark Whetstone. Es una forma de medir la potencia de los
 procesadores. 


\item \textbf{PCI-Express}: Es un nuevo desarrollo del bus PCI apoyado por
Intel, pero se basa en un sistema de comunicación serie mucho más rápido que su
antecesor. La velocidad superior del PCI-Express permitirá reemplazar casi todos
los demás buses, AGP y PCI incluidos. La idea de Intel es tener un solo
controlador PCI-Express comunicándose con todos los dispositivos, en vez de con
el actual sistema de puente norte y puente sur.

\item \textbf{P-ATA}(\textit{Parallel Advanced Technology Attachment}): controla los
 dispositivos de almacenamiento masivo de datos, como los discos duros y ATAPI 
(Advanced Technology Attachment Packet Interface) y además añade dispositivos como 
las unidades CD-ROM.

\item \textbf{Pixel}: El píxel (del inglés picture element, es decir, 
``elemento de la imagen'') es la menor unidad en la que se descompone una 
imagen digital, ya sea una fotografía, un fotograma de vídeo o un gráfico.

\item \textbf{S-ATA}(\textit{Serial Advanced Technology Attachment}): Es una interfaz
para transferencia de datos entre la placa base y algunos dispositivos de
almacenamiento como puede ser el disco duro. Serial ATA sustituye a la
tradicional Parallel ATA o P-ATA (estándar que también se conoce como IDE). El
S-ATA proporciona mayores velocidades, mejor aprovechamiento cuando hay varios
discos, mayor longitud del cable de transmisión de datos y capacidad para
conectar discos en caliente (con la computadora encendida).

\item \textbf{Socket AM2}: Es un zócalo de CPU diseñado por AMD para sus
procesadores de equipos de escritorio. Su lanzamiento se realizó en el segundo
trimestre de 2006, como sustituto del Socket 939. Tiene 940 pines y soporta
memoria DDR2. Los procesadores soportados son las familias Athlon 64 (no en su 
totalidad), Athlon 64X2
y Athlon 64FX.

\end{itemize}

\newpage
\subsection{Anexo: Reportes}

