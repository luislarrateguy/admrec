\section{Descripción del funcionamiento}


\subsection{Arquitectura}

La arquitectura consta de:

\begin{itemize}
  \item Nodos clientes (los cuales son pares entre sí)
  \item Un nodo servidor.
\end{itemize}

Los clientes se van conectando al servidor, indicando su interés llamando a un
servicio web implementado en éste. El servidor, recibida la petición de
ejecución del servicio web, registra la IP de la máquina y el puerto donde
desea recibir notificaciones del servidor. Además proveerá un servicio de
autenticación para el usuario y un de registración (este opcional). Una vez
registrado los datos de conexion, el servidor responde con la lista de personas
que ya están online.

Cuando se quiere establecer una comunicación, el nodo cliente solicita la IP
del otro nodo al servidor. Una vez obtenida la IP, crea un socket de conexión a
esa IP, a través del cual se estarán enviando los objetos de mensajes de texto.

La comunicación entre los pares se cierra al solicitar desconexión. Esto
también es otro servicio web que se encarga de avisar a ambos lados de la
desconexion de la comunicación y cerrar las ventanas de mensajería (y la
liberación de recursos.. sockets, etc).

El envío de mensajes se realizará sin pasar por el servidor, es decir de par a
par. Para esto utilizaremos Remoting.

